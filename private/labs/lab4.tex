\documentclass{article}
\usepackage[margin=1.5cm,bottom=2cm]{geometry}
\usepackage{fancyhdr}
\usepackage{graphicx}
\usepackage{amsmath}
\usepackage{xcolor}
\usepackage{hyperref}
\usepackage[export]{adjustbox}
\pagestyle{fancy}

\hypersetup{colorlinks=true,urlcolor=blue,urlbordercolor=blue}

\begin{document}
\fancyhead[L]{ \includegraphics[width=2cm]{au_logo.png} }
\fancyhead[R]{CPSC 2320: Computational Physics}
\fancyfoot[C]{\thepage}
\vspace*{0cm}
\begin{center}
	{\LARGE \textbf{Lab 4}}\\
	\vspace{0.25cm}
	{\Large Due: Tuesday, October 27}
\end{center}

\section{Vectors}
Repeat the ``Grades'' problem (\href{https://drive.google.com/file/d/1SSxhoP7yIXVp6Xddpqny51gzah2msQGJ/view?usp=sharing}{homework 1}) using a vector to store the grades. All the basic requirements are the same:
\begin{itemize}
	\item User enters grades continuously (until they enter 999)
	\item Grades must be in the range [0,100]
	\item Print both the average \textit{and the standard deviation} (Refer to the instructions in homework 1 for the standard deviation calculation)
\end{itemize}

\section{Unit Testing}
Turn your quadratic formula code from \href{https://drive.google.com/file/d/1qfXRyeUBVmXkNnpzuuMdZz6LJFereyQm/view?usp=sharing}{lab 1, part 1} into a function that accepts the three quadratic coefficients $a, b, $ and $c$ (all doubles). Do not prompt the user for these values, just write a function that accepts them as inputs. There should be no ``cin'' statements anywhere in your program.

Note that in general a quadratic equation will have two solutions. In order to ``return'' these two solutions from our function, we pass two additional {\tt{double}} variables by reference and set their values within the function. To handle cases with no solution, define a global variable: { \tt const double NO\_SOLUTION = -9999; }

Your function should handle the cases described in the lab.
\begin{itemize}
	\item If the solution is complex, $x_1=x_2=$ {\tt NO\_SOLUTION}
	\item If $a\neq0$, $x_1$ and $x_2$ are the two solutions of the quadratic formula
	\item If $a=0$, $x_1=-\frac{c}{b}$, $x_2=$ {\tt NO\_SOLUTION}
\end{itemize}
Now write a unit test to make sure your function works as expected. Make sure you test it for all the different possibilities:
\begin{itemize}
	\item At least one test where $a=0$
	\item At least one test where $\sqrt{b^2-4ac}<0$
	\item At least one test where $\sqrt{b^2-4ac}>0$
	\item At least one test where $\sqrt{b^2-4ac}=0$
\end{itemize}
Run the unit test function from {\tt{main}}

\subsection*{Useful information: Equality comparison with doubles}
The following information will be helpful when completing part 2 of the lab.

Keep in mind that exact equality comparison with doubles is difficult, due to the limited precision of a computer representation of a double. Therefore, the better way to determine if two doubles are equal is to test whether their absolute difference is sufficiently small:
\begin{equation*}
	|x_1-x_2|\leq \epsilon
\end{equation*}
Where $\epsilon$ is small compared to both $x_1$ and $x_2$.

You can implement this in your code in any way you like. One way is to write a new function:\\ {\tt bool are\_equal(double num1, double num2) } which performs the above calculation (you can declare a global variable {\tt const double EPSILON = 1e-5}, or some other very small number, to use for the equality test) and then use this function for your unit tests.
\end{document}