\documentclass{article}
\usepackage[margin=1.5cm,bottom=2cm]{geometry}
\usepackage{fancyhdr}
\usepackage{graphicx}
\usepackage{amsmath}
\usepackage{xcolor}
\usepackage{hyperref}
\usepackage[export]{adjustbox}
\pagestyle{fancy}

\hypersetup{colorlinks=true,urlcolor=blue,urlbordercolor=blue}

\begin{document}
\fancyhead[L]{ \includegraphics[width=2cm]{au_logo.png} }
\fancyhead[R]{CPSC 2320: Computational Physics}
\fancyfoot[C]{\thepage}
\vspace*{0cm}
\begin{center}
	{\LARGE \textbf{Lab 6}}\\
	\vspace{0.25cm}
	{\Large Due: Tuesday, November 10}
\end{center}

A \textit{palindrome} is a word or phrase that is spelled the same way when reversed:
\begin{itemize}
	\item ``racecar''
	\item ``rotator"
	\item ``top spot"
	\item ``no lemon no melon"
\end{itemize}
Write a program that asks the user for a word or phrase and determines whether or not it is a palindrome.
\\
Examples:
\\\\
\texttt{Enter a word or phrase: radar}\\
\texttt{Congratulations! It's a palindrome!}
\\\\
\texttt{Enter a word or phrase: football}\\
\texttt{I am so sorry, but it is not a palindrome :( }
\\\\
\texttt{Enter a word or phrase: taco cat}\\
\texttt{Congratulations! It's a palindrome!}
\end{document}